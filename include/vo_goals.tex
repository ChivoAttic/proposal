%Objetivos
%
%Los volúmenes de datos a gran escala que generan y generarán los observatorios
%astronómicos actuales y futuros en Chile, han generado nuevas necesidades que
%sirven como oportunidades de desarrollo de nuevas herramientas y técnicas de
%análisis de datos.
%
%Para tener una idea general de la cantidad de datos que requerirán ser
%clasificados y procesados, podemos tomar en cuenta el proyecto ALMA,
%inaugurado hace un par de semanas, el cual cuando se encuentre completamente
%operativo (con todo su conjunto de antenas), se generarán más de 1 TB de datos
%por día de observación.
%
%La manipulación de altos volúmenes de datos generan complicaciones en los
%siguientes aspectos:
%
%    Almacenamiento, es necesario tener un centro de procesamiento de datos con
%la capacidad de almacenamiento suficiente acorde a las necesidades de consumo
%de dichos datos, sin dejar de lado el espacio físico que dichos equipos
%necesitarán y la architectura detrás del almacenamiento en sí.
%
%    Acceso, se deben establecer mecánicas y normativas de accesos para
%cualquier persona, ya sean astrónomos o no, lo cual requiere de un sistema que
%permita acceder a dicho sistema desde cualquier lugar, en el presente proyecto
%se ha decidido un sistema web es un mecanismo adecuado para suplir la presente
%necesidad.
%
%    Procesamiento, el procesamiento de datos es un área particular de la
%informática, la cual conlleva entender tanto la naturaleza como la estructura
%de los datos, como también las herramientas y técnicas que se pueden utilizar
%para llevarlo a cabo. Al tener grandes volúmenes de datos, el procesamiento a
%realizar, ya sean correcciones, calibraciones, análisis, etc., exige más tiempo
%del habitual y por supuesto, más recursos computacionales, elementos que un
%usuario no posee. Por lo que un cluster computacional, puede ser una
%herramienta útil para facilitar esta labor.
%
%Los puntos anteriormente detallados pertenecen a las motivaciones principales
%por las cuales actualmente en Chile, se está desarrollando un proyecto que
%busca crear una plataforma de procesamiento de datos de gran escala, la cual
%compromete como uno de sus entregables un Observatorio Virtual, del que se
%espera que garantice rapidez y eficiencia, tanto al acceso de información
%existente y servicios astronómicos, como en el análisis de dicha información.

\section{Goals}

The current and future large-scale date generated by the astronomical observatories
placed in Chile, have created new needs which serve as an opportunity
for the development of new data analysis tools and techniques.


To have a general idea of the amount of data that will need to be classified and
processed, we can consider the ALMA project, inaugurated a couple of weeks ago,
which when fully operational (with all the complete array) will generate over 1TB
of data per observation day.

The handling of high data volumes generate complications in the following
issues:

\begin{itemize}
    \item \emph{\textbf{Storage}}, is necessary to have data center
        capable of storage according the data consumption needs,
        without ignoring the physical space of the equipment and
        the architecture behind the storage procedure.
    \item \emph{\textbf{Access}}, the mechanical and regulations must be
        established for anyone, whether or not astronomers, which
        requires a system that allows access from anywhere.
        In this project it has been decided that a web system is an
        appropriate mechanism to supply this requirement.
    \item \emph{\textbf{Processing}}, data processing is a particular area
        of computer science, which means understanding the nature and the
        structure of the data, as well as the tools and techniques
        that can be used to carry it out. Having large volumes of data,
        the processing to accomplish, whether corrections, calibrations,
        analysis, etc., requires more time than usual and of course,
        more computational resources, items that a user does not have.
        So a computational cluster, can be an useful tool to facilitate
        this work.
\end{itemize}

The points detailed above, belongs to the main motivations of why Chile is currently
developing a project which aims to create a platform for large-scale data processing,
which promises like one of their deliverable a Virtual Observatory,
which is expected to ensure speed an efficiency for the existing information access
and astronomical services, as in the analysis of such information.
